\documentclass[../bachelor_paper.tex]{subfiles}
\graphicspath{{\subfix{images/}}}
\begin{document}

\chapter{Theory}
    \label{ch:theory}

According to Wikipedia, a benchmark in computing is \emph{the act of running a computer program, a set of programs, or other operations, in order to assess the relative performance of an object, normally by running a number of standard tests and trials against it. The term benchmark is also commonly utilized for the purposes of elaborately designed benchmarking programs themselves.} \cite{BenchmarkComputing2021} \\
We will use the work \emph{benchmark} to refer to the latter, a (set of) program(s) constructed or assembled to gage the performance of an artifact, in order to gain insight on how said artifact might perform in a real world application compared to other artifacts. It is important to note however, that interpolating hard performance numbers of a target program from the sole run of a benchmark is almost impossible since different programs have different characteristics. This problem will be the main focus of this paper.\\

\section*{Two Benchmarks (and then a meteor strikes)}
We will start this section by mentioning the two in scientific circles ost well known benchmark suits, the \emph{SPEC suit} and \emph{Coremark}.

\subsection*{SPEC suite}
\subsection*{Coremark}

% Render bibliograhy and acronyms if rendered standalone
\isstandalone
\bibliographystyle{IEEEtran}
\bibliography{bibliography}
\subfile{abbreviations.tex}
\fi

\end{document}