%!TEX spellcheck
%!TEX root = ../bachelor_paper.tex
\chapter{How To}
    \label{app:how_to}
    
In order to build Datalynx for yourself, clone the modified Pulpissimo repository. This was tested on Ubuntu 16.04 with Xilinx Vivado 2018.3. Make sure to complete all steps for building Pulpissimo as described in the top level Readme file. After the \texttt{update-ips} script has been executed, navigate to \texttt{fpga/pulpissimo-zedboard} and execute the \texttt{patch-ips.sh} script to apply the necessary patches to the IPs provided by Pulpissimo. This will add the modifications for your design to contain Datalynx. Do not do this if you intend to build for platforms other than Zedboard. No other hardware is supported at this time! After that you can simulate the design and compile a bitstream as normally.

After bitstream generation, navigate to \texttt{fpga/pulpissimo-zedboard/sw} and call \texttt{make} to build the embedded Linux image containing the modified Pulpissimo design. If Xilinx Bootgen returns an error upon first stage bootloader generation, call make again from within Xilinx \texttt{xsct}. After finishing the make process, the final files can be flashed onto an SD-card using the \texttt{install.sh} script. Be careful as this deletes all contents preset on the selected medium! On Ubuntu 16.04 it is possible to encounter issues when trying to boot from the generated SD-card. The simplest solution is copying the final files and flash script to a newer version of Ubuntu.

In order to activate the Datalynx extension during runtime, modify your workload to change the value of the register address \texttt{0x7A2} first to \texttt{0b110} to reset the counters and then to \texttt{0b001} to start counting. After the workload has finished, set the register to \texttt{0b010} to signal completion to Linux. 

To log data on Linux, open a remote session to the ARM-machine, either by serial console or SSH. Call \texttt{./delynx <workload>} from the \texttt{/root/} folder prior to program execution. As soon as Pulpissimo signals start of workload execution by setting \texttt{0x7A2} to \texttt{0b001}, \texttt{delynx} will start logging data into the file \texttt{<workload>.csv}. CSV column layout can be seen in Table \ref{tab:howto/datalayout}. The second to last row in the file contains the total count of each datapoint while the last row indicates the final state of the overflow vector. Should this line contains anything other than a 0, data is to be treated as invalid.

\begin{table}
    \centering
    \caption{Column layout for CSV files generated by \texttt{delynx}; top to bottom is left to right}
    \begin{tabular}{lp{0.7\textwidth}}
        \textbf{Datapoint}  & \textbf{Description} \\
        \hline
        Time            & Time passed since start of workload execution (in ms) \\
        Instructions    & Total count of instructions executed (equal to $2^{\texttt{NLYNX\_SECTION\_SIZE}}$ in in-flight data) \\
        Loads           & Number of load instructions encountered \\
        Stores          & Number of store instructions encountered \\
        ALU other       & Cycles ALU has been active and no other instructions have been executed. Has been replaced by calculating the remainder of unassigned instructions due to the RI5CY ALU being active even when other instructions are executed. \\
        Multiplications & Number of multiplication instructions encountered \\
        Branch          & Number of branch instructions encountered \\
        Branches taken  & Number of taken branches \\
        FPU             & Number of FPU instructions encountered \\
        Jump            & Number of jump instructions encountered \\
        \ac{HWL} init   & Number of \ac{HWL} initializations encountered \\
        \ac{HWL} jump   & Number of \ac{HWL} jumps performed \\
        Instruction fetch   & Number of instructions fetched \\
        Cycles wasted   & Number of cycles wasted. This counter is not increased when clock gating is active! \\
    \end{tabular}
    \label{tab:howto/datalayout}
\end{table}