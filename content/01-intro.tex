\documentclass[../bachelor_paper.tex]{subfiles}
\graphicspath{{\subfix{images/}}}
\begin{document}

\chapter{Introduction}
    \label{ch:intro}

Just having bought a new processor or computer, many people want to know how fast it can run first to evaluate their new toy. It is an inherent desire to put them to the test like trying out a new car or playing the first riff on a guitar. But in microprocessors, knowing just how fast it can go is an integral part of the development process of the core and surrounding systems itself as well as any product using said microprocessor or core. How can we say the controller of a actuator doesn't get overwhelmed and the plane loses control?\\ 
The process of evaluating just how fast the micro processing unit or \acs{MPU} can run is called a benchmark. Over the decades, creating these benchmarks gained more and more scientific attention, evolving from the attempt to hit as many common workloads as possible to analyzing similarity between the different programs employed by a benchmark suit. In this thesis we will start by looking at different attempts to evaluate similarity of programs in a single benchmark suit and the ``quality'' of a benchmark suit in general. Second we will look at several benchmarks relevant mostly to the x86 \acs{ISA} as well as Coremark\cite{gal-onExploringCoremarkBenchmark2012}, a platform agnostic benchmark suitable for smaller cores which are potentially not focused on Linux-like \acs{OS} support.

\isstandalone
\bibliographystyle{IEEEtran}
\bibliography{bibliography}
\subfile{abbreviations.tex}
\fi

\end{document}