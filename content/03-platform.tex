%!TEX spellcheck
%!TEX root = ../bachelor_paper.tex
\documentclass[../bachelor_paper.tex]{subfiles}
\graphicspath{{\subfix{images/}}}
\begin{document}

\chapter{Platform}
    \label{ch:plat}

\section{Instruction Set Architecture}
We chose the open RISC-V \ac{ISA} as our target. RISC-V has seen a surge in popularity in recent years, as its usage is completely free and open. This has lead to a flourishing ecosystem of open IPs from different institutions with different goals in mind. The architecture is a basic RISC like load/store architecture featuring an unprivileged and privileged specification \cite{RISCVInstructionSet2022}.

\section{Base IP}
We base our design on the Pulpissimo microcontroller IP provided by the ETH Z\"urich, and University of Bologna \cite{schiavoneQuentinUltraLowPowerPULPissimo2018}. The core used in the design is the CV32E40P, formerly known and here (for simplicity) referred to as RI5CY. RI5CY is a 32-bit core with a single-issue, in-order, 4-stage pipeline implementing the RV32IM[F]C instruction set. Floating point support can be enabled and disabled ([F]) depending on requirements towards the core. We will have floating point support enabled as dedicated hardware \acp{FPU} have become rather common even in embedded \acp{SOC}. 

\subsection{Pipeline}
As seen in figure \ref{fig:plat/base/blockdia}, RI5CY consists of 4 pipeline stages. Stage one is an instruction fetch stage using a 4 entry prefetch buffer as well as a \ac{HWL} controller, which we will go into detail about in section \todo{add section reference}. Stage 2 is the instruction decode stage, consisting of a RISC-V decoder \todo{does this feed back to the hwl controller directly?}. Stage 3 is the execute stage featuring an \ac{ALU} for arithmetic and logic operations except multiplication, as well as a multi-cycle multiplication unit which can also perform multiply accumulate operations \todo{check this in hardware}. Additionally, if enabled, the exec stage also contains the \ac{FPU} which we will go further into in section \todo{add reference}. The 4th and final pipeline stage contains the load/store unit as well as the register write-back.

\begin{figure}
    \centering
    %\includesvg[width=0.6\columnwidth]{ri5cy_blockdiagram}
    \includegraphics[width=0.6\columnwidth]{placeholder}
    \caption{Block diagram of the RI5Cy pipeline}
    \label{fig:plat/base/blockdia}
\end{figure}


% Render bibliograhy and acronyms if rendered standalone
\isstandalone
\bibliographystyle{IEEEtran}
\bibliography{bibliography}
\subfile{abbreviations.tex}
\fi

\end{document} 
