%!TEX spellcheck
%!TEX root = ../bachelor_paper.tex
\documentclass[../bachelor_paper.tex]{subfiles}
\graphicspath{{\subfix{images/}}}
\begin{document}

\chapter{Conclusions}
    \label{ch:conc}
    
We have presented Datalynx, a new way to gather embedded workload performance telemetry targeting a feature aware approach to program similarity. We demonstrated the capabilities of this framework using industry standard benchmarks and have thus laid the groundwork for more optimized implementations. While our naive implementation is rather hardware resource intensive, we have been able to provide a proof of concept that may be modified to make sense outside academic applications. Datalynx was implemented as an add-on to the open source RI5CY core. We were able to gather performance data from the benchmarks validated on and demonstrate the logging capabilities of the system. We additionally calculated the correlation coefficient between the metrics collected and were able to show strong correlation between some of them, marking them for removal in future works. We saw very strong correlation between the mean \ac{HWL} initialization distance and mean \ac{HWL} iteration count but were not able to completely rule out independence of the variables.

Lastly, we have provided performance profiles for the applications tested on Datalynx were able to identify 3 different behavioral categories within them.

% Render bibliograhy and acronyms if rendered standalone
\isstandalone
\bibliographystyle{IEEEtran}
\bibliography{bibliography}
\subfile{abbreviations.tex}
\fi

\end{document} 
