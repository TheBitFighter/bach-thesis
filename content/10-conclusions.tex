%!TEX spellcheck
%!TEX root = ../bachelor_paper.tex
\documentclass[../bachelor_paper.tex]{subfiles}
\graphicspath{{\subfix{images/}}}
\begin{document}

\chapter{Future Work and Conclusions}
    \label{ch:future-conc}
Future work might desire to develop a method of selecting selecting a subset of benchmarks for a given workload. This would allow us to simply run this reduced set and garner the metrics needed to accurately predict target workload performance instead of blindly running benchmarks, thus judging overall system performance when only certain aspects might be needed. We plan to run a real world workload against our modified system to test this hypothesis. This tool would need to calculate weighting coefficients to correctly estimate the performance of the target workload. Given the overlap of the metrics proposed in this paper, this future effort might choose to remove one or more metrics from its suite.

Additionally, further work might be interested in adapting the metrics collected to a more general processor model. The data points implemented in this thesis have been simplified vastly given the minimalist nature of the RI5CY memory system. For example, expanding to include metrics such as branch transition rate \cite{haungsBranchTransitionRate2000} instead of just branch direction would benefit similarity measurement on platforms employing more complex branch prediction schemes. Further look into the relevance of temporal changes behavior might also be of interest. If these changes prove to be relevant beyond a certain degree, a scheme akin to \cite{joshiDistillingEssenceProprietary2008} might be of more use than measuring program similarity, as the search space for relevant stand in workloads would increase even further. These possible expansions were outside of the scope of this thesis but would be imperative to rule out implementation reliant metrics and other influences.


\section{Conclusions}
    \label{sec:fut/conc}
    
We have presented Datalynx, a new way to gather embedded workload performance telemetry using a feature-aware approach to program similarity. We demonstrated the capabilities of this framework using industry standard benchmarks and have thus laid the groundwork for more optimized implementations. While our naive implementation is rather hardware resource intensive, we have been able to provide a proof of concept that may be modified to make sense outside academic applications. Datalynx was implemented as an add-on to the open-source RI5CY core. We were able to gather performance data from the benchmarks validated on and demonstrate the logging capabilities of the system. We additionally calculated the correlation coefficient between the metrics collected, which showed high correlation between some of them. At the same time we were also able to show that this correlation does not show the whole image as counterexamples could already be shown in our rather limited data. We saw very strong correlation between the mean \ac{HWL} initialization distance and mean \ac{HWL} iteration count but were not able to completely rule out independence of the variables either.

Lastly, we have provided performance profiles for the applications tested on Datalynx.

% Render bibliograhy and acronyms if rendered standalone
\isstandalone
\bibliographystyle{IEEEtran}
\bibliography{bibliography}
\subfile{abbreviations.tex}
\fi

\end{document} 
