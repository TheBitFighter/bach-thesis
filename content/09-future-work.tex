%!TEX spellcheck
%!TEX root = ../bachelor_paper.tex
\documentclass[../bachelor_paper.tex]{subfiles}
\graphicspath{{\subfix{images/}}}
\begin{document}

\chapter{Future Work}
    \label{ch:future}
Future work might desire to develop a method of selecting selecting a subset of benchmarks for a given workload. This would allow us to simply run this reduced set and garner the metrics needed to accurately predict target workload performance instead of blindly running benchmarks, thus judging overall system performance when only certain aspects might be needed. We plan to run a real world workload against our modified system to test this hypothesis. This tool would need to calculate weighting coefficients to correctly estimate the performance of the target workload. Given the overlap of the metrics proposed in this paper, this future effort might choose to remove one or more metrics from its suite.

Additionally, further work might be interested in adapting the metrics collected to a more general processor model. The data points implemented in this thesis have been simplified vastly given the minimalist nature of the RI5CY memory system. For example, expanding to include metrics such as branch transition rate \cite{haungsBranchTransitionRate2000} instead of just branch direction would benefit similarity measurement on platforms employing more complex branch prediction schemes. These possible expansions were outside of the scope of this thesis but would be imperative to rule out implementation reliant metrics.


% Render bibliograhy and acronyms if rendered standalone
\isstandalone
\bibliographystyle{IEEEtran}
\bibliography{bibliography}
\subfile{abbreviations.tex}
\fi

\end{document} 
