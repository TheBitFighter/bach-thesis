Die Auswahl des korrekten Benchmarks zur Ermittlung von Task-Fitness oder Task-spezifischer Hardwareentwicklung ist von essentieller Bedeutung. Die Messung von Programmvergleichbarkeit kann bei der Auswahl des richtigen Benchmarks zur Optimierung eines bestimmten Tasks helfen. Wir präsentieren Datalynx, ein Hardware-basiertes Framework zur Messung von Programmvergleichbarkeit, welches die nötigen Tools direkt in Hardware zur Verfügung stellt. Wir besprechen zuerst vorangegangene Lösungswege zur Messung von Programmvergleichbarkeit und definieren daraus eine Liste von gemessenen Features für diese Implementation. Datalynx wurde als Add-On für den open-source RISC-V Kern RI5CY implementiert und im Pulpissimo Mikrocontroller auf der Xilinx Zynq-7000 Plattform getestet. Diese Implementation erlaubt uns Telemetrie in-flight zu ermitteln, wodurch zeitaufwendige Simulation als auch Einschränkungen der messbaren Daten sowie Performanceauswirkungen bei Messung auf unmodifizierter Hardware wegfallen. Die Kernmodifikation \emph{Enlynx} zeigt nur leichte Auswirkungen auf die Leistungsaufnahme des Kerns, es handelt sich jedoch um einen simplifizierten Proof-Of-Concept. Wir waren mit dieser Implementation in der Lage, die Funktionsfähigkeit einer solchen Lösung anhand von als Industriestandard gehandelten Benchmarks zu demonstrieren und die getesteten Benchmarks in drei Gruppen anhand ihres zeitlichen Verhaltens einzuteilen.