% Copyright (C) 2014-2020 by Thomas Auzinger <thomas@auzinger.name>

\documentclass[draft,final]{vutinfth} % Remove option 'final' to obtain debug information.

% Load packages to allow in- and output of non-ASCII characters.
\usepackage{lmodern}        % Use an extension of the original Computer Modern font to minimize the use of bitmapped letters.
\usepackage[T1]{fontenc}    % Determines font encoding of the output. Font packages have to be included before this line.
\usepackage[utf8]{inputenc} % Determines encoding of the input. All input files have to use UTF8 encoding.

% Extended LaTeX functionality is enables by including packages with \usepackage{...}.
\usepackage{amsmath}    % Extended typesetting of mathematical expression.
\usepackage{amssymb}    % Provides a multitude of mathematical symbols.
\usepackage{mathtools}  % Further extensions of mathematical typesetting.
\usepackage{microtype}  % Small-scale typographic enhancements.
\usepackage[inline]{enumitem} % User control over the layout of lists (itemize, enumerate, description).
\usepackage{multirow}   % Allows table elements to span several rows.
\usepackage{booktabs}   % Improves the typesettings of tables.
\usepackage{subcaption} % Allows the use of subfigures and enables their referencing.
\usepackage[ruled,linesnumbered,algochapter]{algorithm2e} % Enables the writing of pseudo code.
\usepackage[usenames,dvipsnames,table]{xcolor} % Allows the definition and use of colors. This package has to be included before tikz.
\usepackage{nag}       % Issues warnings when best practices in writing LaTeX documents are violated.
\usepackage{todonotes} % Provides tooltip-like todo notes.

%%%%%%%% CUSTOM STUFF %%%%%%%%

% Load subfile package at the end to enable per fiel compilation
%\graphicspath{{img/}}
\usepackage{xr}
\usepackage{subfiles}
\externaldocument[]{\subfix{bachelor_paper.tex}}

%% Align positive and negative numbers in tables
\usepackage{siunitx}
%%% Enable text rotation
\usepackage{rotating}
%% Better tables i guess
\usepackage{tabularx}
%% Package for graphs
\usepackage{epstopdf}
%%% Package to simplify block quotes
\usepackage{csquotes}
%%% Package to maintain and use acronyms %%%
\usepackage[printonlyused]{acronym}
%% Float barriers
\usepackage{placeins}

\definecolor{tuwblue}{RGB}{00,102,153}
\usepackage[colorlinks=true,
    linkcolor=black,
    citecolor=black,
    urlcolor=tuwblue,
    bookmarks=true,
    bookmarksopen=true,
    bookmarksopenlevel=3,
    plainpages=false,
    pdfpagelabels=true]{hyperref}

%%%%%%%%%%%%%%%%%%%%%%%%%%%%%%%

%\usepackage{hyperref}  % Enables cross linking in the electronic document version. This package has to be included second to last.


% Define convenience functions to use the author name and the thesis title in the PDF document properties.
\newcommand{\authorname}{Fabian Philipp Posch} % The author name without titles.
\newcommand{\thesistitle}{Datalynx} % The title of the thesis. The English version should be used, if it exists.

% Set PDF document properties
\hypersetup{
    pdfpagelayout   = TwoPageRight,           % How the document is shown in PDF viewers (optional).
    linkbordercolor = {Melon},                % The color of the borders of boxes around crosslinks (optional).
    pdfauthor       = {\authorname},          % The author's name in the document properties (optional).
    pdftitle        = {\thesistitle},         % The document's title in the document properties (optional).
    pdfsubject      = {Subject},              % The document's subject in the document properties (optional).
    pdfkeywords     = {a, list, of, keywords} % The document's keywords in the document properties (optional).
}

\setpnumwidth{2.5em}        % Avoid overfull hboxes in the table of contents (see memoir manual).
\setsecnumdepth{subsection} % Enumerate subsections.

\nonzeroparskip             % Create space between paragraphs (optional).
\setlength{\parindent}{0pt} % Remove paragraph identation (optional).

\makeindex      % Use an optional index.
%\glstocfalse   % Remove the glossaries from the table of contents.

% Set persons with 4 arguments:
%  {title before name}{name}{title after name}{gender}
%  where both titles are optional (i.e. can be given as empty brackets {}).
\setauthor{}{\authorname}{}{male}
\setadvisor{Ao.Univ.Prof. Dipl.-Ing. Dr.techn.}{Andreas Steininger}{}{male}

% For bachelor and master theses:
\setfirstassistant{Univ.Ass. Dipl.-Ing.}{Stefan Tauner}{}{male}
%\setsecondassistant{Pretitle}{Forename Surname}{Posttitle}{male}
%\setthirdassistant{Pretitle}{Forename Surname}{Posttitle}{male}

% For dissertations:
%\setfirstreviewer{Pretitle}{Forename Surname}{Posttitle}{male}
%\setsecondreviewer{Pretitle}{Forename Surname}{Posttitle}{male}

% For dissertations at the PhD School and optionally for dissertations:
%\setsecondadvisor{Pretitle}{Forename Surname}{Posttitle}{male} % Comment to remove.

% Required data.
\setregnumber{01456625}
\setdate{24}{03}{2022} % Set date with 3 arguments: {day}{month}{year}.
\settitle{\thesistitle}{Titel der Arbeit} % Sets English and German version of the title (both can be English or German). If your title contains commas, enclose it with additional curvy brackets (i.e., {{your title}}) or define it as a macro as done with \thesistitle.
\setsubtitle{An in-hardware framework for assessment of program similarity}{Optionaler Untertitel der Arbeit} % Sets English and German version of the subtitle (both can be English or German).

% Select the thesis type: bachelor / master / doctor / phd-school.
% Bachelor:
\setthesis{bachelor}
%
% Master:
%\setthesis{master}
%\setmasterdegree{dipl.} % dipl. / rer.nat. / rer.soc.oec. / master
%
% Doctor:
%\setthesis{doctor}
%\setdoctordegree{rer.soc.oec.}% rer.nat. / techn. / rer.soc.oec.
%
% Doctor at the PhD School
%\setthesis{phd-school} % Deactivate non-English title pages (see below)

% For bachelor and master:
\setcurriculum{Computer Engineering}{Technische Informatik} % Sets the English and German name of the curriculum.

% For dissertations at the PhD School:
%\setfirstreviewerdata{Affiliation, Country}
%\setsecondreviewerdata{Affiliation, Country}


\begin{document}

\frontmatter % Switches to roman numbering.
% The structure of the thesis has to conform to the guidelines at
%  https://informatics.tuwien.ac.at/study-services

%\addtitlepage{naustrian} % German title page (not for dissertations at the PhD School).
\addtitlepage{english} % English title page.
\addstatementpage

\begin{acknowledgements*}
My parents, whose patience I have been able to stretch generously as I worked on this very lengthy project. As well as my supervisor Stefan Tauner for removing roadblocks and sharing opinions about FPGA toolchains.

Lieven Eeckhout, Ajay Joshi, and Aashish Phansalkar, whose names I kept running into while digging through mountains of papers trying to understand what features of a program actually influence similarity. They have built much of the groundwork this thesis stands on.

I would also like to thank my good friend Charlie Long for providing the name \emph{Datalynx} through a misread on my end.\\
As well as him, Taylor Wilkins, and Ashley Joy for helping me English good.

Finally I would like to thank F.D.C. Willard for his insights into related topics and for inspiring me to broaden my horizons.
\end{acknowledgements*}

\begin{kurzfassung}
\documentclass[../bachelor_paper.tex]{subfiles}
\graphicspath{{\subfix{images/}}}
\begin{document}

\selectlanguage{ngerman}
\currentpdfbookmark{Kurzfassung}{chap:kurzfassung}
\section*{Kurzfassung}\thispagestyle{empty}
    \label{chap:kurzfassung}

    Text hier einfügen \dots
    
    \vfill
% Please insert 3-5 German keywords that characterize the thesis:
\paragraph*{Schlagwörter:} Schlagwort 1, Schlagwort 2, \dots
\selectlanguage{english}

\end{document}
\end{kurzfassung}

\begin{abstract}
Choosing the correct benchmarks for task fitness testing or task specific hardware development is crucial. Measuring program similarity can help in selecting the right benchmark for the right task to be optimized. We present Datalynx, a hardware based framework built for measuring workload similarity, providing the tools necessary for recording data directly in hardware. We first discuss and compare previous approaches to measuring program and benchmark similarity and define a set of features measured for this implementation. Datalynx was implemented as an add-on to the open source RISC-V implementation RI5CY and tested within the Pulpissimo micro controller on the Xilinx Zynq-7000 platform. This implementation allows us to gather telemetry in flight in real time, eliminating the need for time consuming simulation as well as overcoming data availability restrictions or performance impacts when measuring on unmodified hardware. While being a proof of concept, the actual core modifications named \emph{Enlynx} show little power impact. We were able to demonstrate a functioning implementation of this approach as well as gather performance profiles from several industry standard benchmarks and categorize them into three main behavioral groups.
\end{abstract}

% Select the language of the thesis, e.g., english or naustrian.
\selectlanguage{english}

% Add a table of contents (toc).
\tableofcontents % Starred version, i.e., \tableofcontents*, removes the self-entry.

% Switch to arabic numbering and start the enumeration of chapters in the table of content.
\mainmatter
    % Fix bibliography stuff with subfiles
    \def\isstandalone{\iffalse}

    %%%%% MAIN CONTENT %%%%%%%%%%%%%%%%%%%%%%%%%%%%%%%%%%%%%%%%%%%%%%%%%%%%%%%%% Add content here
    \subfile{content/01-intro.tex}
    \subfile{content/02-problem.tex}
    \subfile{content/03-platform.tex}
    \subfile{content/04-architecture.tex}
    \subfile{content/05-performance.tex}
    \subfile{content/06-benchmarks.tex}
    \subfile{content/08-results.tex}
    \subfile{content/09-future-work.tex}
    \subfile{content/10-conclusions.tex}

    %\subfile{content/04-results.tex}
    %\subfile{content/05-conclusion.tex}
    %%%%%%%%%%%%%%%%%%%%%%%%%%%%%%%%%%%%%%%%%%%%%%%%%%%%%%%%%%%%%%%%%%%%%%%%%%%%

\backmatter

% Use an optional list of figures.
\listoffigures % Starred version, i.e., \listoffigures*, removes the toc entry.

% Use an optional list of tables.
\cleardoublepage % Start list of tables on the next empty right hand page.
\listoftables % Starred version, i.e., \listoftables*, removes the toc entry.

% Use an optional list of alogrithms.
%\listofalgorithms
%\addcontentsline{toc}{chapter}{List of Algorithms}

% Add an index.
\printindex

% Add a glossary.
\subfile{abbreviations.tex}

% Add a bibliography.
\bibliographystyle{alpha}
\bibliography{bibliography}

%%%%% APPENDIX %%%%%%%%%%%%%%%%%%%%%%%%%%%%%%%%%%%%%%%%%%%%%%%%%%%%%%%%%%%%%
\mainmatter
\appendix
\setcounter{page}{57}
\subfile{content/A-how_to.tex}
\subfile{content/B-graphs.tex}
%%%%%%%%%%%%%%%%%%%%%%%%%%%%%%%%%%%%%%%%%%%%%%%%%%%%%%%%%%%%%%%%%%%%%%%%%%%%

\end{document}